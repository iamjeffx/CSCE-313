\documentclass[12pt]{article}

\newcommand\tab[1][1cm]{\hspace*{#1}}

\usepackage{times}
\usepackage{amsmath}
\usepackage{latexsym}
\usepackage{fullpage}
\usepackage{graphicx}
\usepackage{amsfonts}
\usepackage[dvipsnames]{xcolor}

\graphicspath{ {./images/} }

\newcommand{\NOT}{\neg}
\newcommand{\AND}{\wedge}
\newcommand{\OR}{\vee}
\newcommand{\XOR}{\oplus}
\newcommand{\IMPLIES}{\rightarrow}
\newcommand{\IFF}{\leftrightarrow}
\newcommand{\E}{\exists}
\newcommand{\A}{\forall}

\setlength{\parskip}{.1in}

\renewcommand{\baselinestretch}{1.1}

\begin{document}

\begin{center}

{\bf
CSCE 313\\
Quiz 3\\
Jeffrey Xu\\
527008162\\
10/20/20\\
}

\end{center}

{\bf 1. [20 pts]} Assume the following processes A, B, C are loaded in memory of a system that uses multiprogramming. These processes have 13, 7 and 11 instructions respectively, Also assume that the dispatcher lives at address 100 in memory and spans 4 instructiors (i.e. 100-103). The following table shows only instruction addresses in the memory with I/O requests labelled, along with the duration of these I/O operations in terms of CPU instructions. Although I/O operations do not take CPU instructions, the duration means that the I/O operations will finish by the time the corresponding number of CPU instructions execute. Please draw a trace of these 3 processes running together in the CPU using Shortest Remaining Time First (SRTF) with preemption and no timer (i.e., we are not running a timer that is used by time sharing or round-robin). Recall that for SRTF, you only need to decide based on the next CPU burst, not the entire remaining time of the process. You can skip the first invocation of the dispatcher to decide the first process to run in the CPU.

\begin{center}
\begin{tabular}{| c | c | c |}
\hline
{\bf Process A} & {\bf Process B} & {\bf Process C}\\
\hline\hline
5000 & 8000 & 12000\\
\hline
5001 & 8001 & 12001 (I/O, takes 3 ins.)\\
\hline
5002 & 8002 & 12002\\
\hline
5003 & 8003 (I/O, takes 7 ins.) & 12003\\
\hline
5004 (I/O, takes 6 ins.) & 8004 & 12004\\
\hline
5005 (I/O, takes 4 ins.) & 8005 & 12005\\
\hline
5006 & 8006 & 12006 (I/O, takes 1 ins.)\\
\hline
5007 & & 12007\\
\hline
5008 (I/O, takes 5 ins.) & & 12008 (I/O, takes 2 ins.)\\
\hline
5009 & & 12009\\
\hline
5010 & & 12010\\
\hline
5011 & &\\
\hline
5012 & &\\
\hline
\end{tabular}
\end{center}

{\bf Solution:}\\

We must decide which process to run next based on each processes CPU bursts from previous CPU cycles. The shortest CPU remaining time will be the number of instructions till the next I/O instruction. Also, since we are using preemptive STRF, anytime a new process gets put into the ready queue, we must do a context switch. We also assume that A, B and C all are loaded at the beginning. We also are doing a context switch everytime a process goes from the blocked queue to the ready queue. 

\noindent
1. 12000\\
2. 12001\\
----I/O----\\
3. 100\\
4. 101\\
5. 102\\
6. 103\\
7. 8000\\
8. 8001\\
9. 8002\\
10. 8003\\
----I/O----\\
11. 100\\
12. 101\\
13. 102\\
14. 103\\
15. 12002\\
16. 12003\\
17. 12004\\
----New Process in----\\
18. 100\\
19. 101\\
20. 102\\
21. 103\\
22. 12005\\
23. 12006\\
----I/O----\\
24. 100\\
25. 101\\
26. 102\\
27. 103\\
28. 12007\\
29. 12008\\
----I/O----\\
30. 100\\
31. 101\\
32. 102\\
33. 103\\
34. 12009\\
35. 12010\\
----I/O----\\
36. 100\\
37. 101\\
38. 102\\
39. 103\\
40. 8004\\
41. 8005\\
42. 8006\\
----I/O----\\
43. 100\\
44. 101\\
45. 102\\
46. 103\\
47. 5000\\
48. 5001\\
49. 5002\\
50. 5003\\
51. 5004\\
----I/O----\\
52. 100\\
53. 101\\
54. 102\\
55. 103\\
56. No-op\\
57. No-op\\
----New Process in----\\
58. 100\\
59. 101\\
60. 102\\
61. 103\\
62. 5005\\
----I/O----\\
63. 100\\
64. 101\\
65. 102\\
66. 103\\
----New Process in----\\
67. 100\\
68. 101\\
69. 102\\
70. 103\\
71. 5006\\
72. 5007\\
73. 5008\\
----I/O----\\
74. 100\\
75. 101\\
76. 102\\
77. 103\\
78. No-op\\
----New process in----\\
79. 100\\
80. 101\\
81. 102\\
82. 103\\
83. 5009\\
84. 5010\\
85. 5011\\
86. 5012\\

{\bf 2 [5 pts]} The following is the Producer function in a BoundedBuffer implementation. What is the purpose of the mutex in the following? Can we do without the mutex? In what circumstances? 

Producer(item) \{\\
\tab emptySlots.P();\\
\tab mutex.P();\\
\tab enqueue(item);\\
\tab mutex.V();\\
\tab fullSlots.V();\\
\}

{\bf Solution:}\\

The mutex in this function protects the critical section of enqueuing an item onto the queue for the BoundedBuffer. It makes sure that the machine is free to operate on the buffer and makes sure that there are no race conditions. We could do without the mutex only in the case that there is one producer. If there are multiple producers, then you must need a mutex to prevent race conditions. 

{\bf 3 [5 pts]} The following is the Producer function in a BoundedBuffer implementation. Can we change the order of the first 2 lines? Why or why not?

Producer(item) \{\\
\tab emptySlots.P();\\
\tab mutex.P();\\
\tab enqueue(item);\\
\tab mutex.V();\\
\tab fullSlots.V();\\
\}

{\bf Solution:}\\

You cannot change the order of the first two lines because you may cause a deadlock. If the producer locks the mutex, then no other thread can access the CPU. Then, if the producer needs for emptySlots to increment, it must depend on a consumer. However a consumer cannot access the queue since the producer locked the mutex so a deadlock would occur between the producer and consumer. 

{\bf 4 [10 pts]} If we run 5 instances of ThreadA() and 1 of ThreadB(), what can be the maximum number of threads active simultaneously in the Critical Section? The mutex is initially unlocked. Note that ThreadB() is buggy and mistakenly unlocks the mutex first instead of locking first. Explain your answer. 

\noindent ThreadA() \{\\
\tab muetx.P()\\
\tab /* Start Critical Section */\\
\tab ......\\
\tab /* End Critical Section */\\
\}\\

\noindent ThreadB() \{ \\
\tab mutex.V() \\
\tab /* Start Critical Section */\\
\tab .....\\
\tab /* End Critical Section */ \\
\tab mutex.V()\\
\}

{\bf Solution:}\\

The maximum number of threads in the critical section is three. Assume that we run an instance of ThreadA first. This instance will lock the mutex and start running the critical section. Now, assume that we encounter a context switch in the middle of the critical section of this instance of ThreadA and now start running an instance of ThreadB. Now ThreadB will unlock the mutex and start running the critical section making it 2 threads in the critical section. Then, assume that a context switch occurs in ThreadB and we run a new instance of ThreadA. Now this new instance of ThreadA will lock the mutex and start running the critical section. This gives us 3 individual threads running the critical section simultaneously. We see that if we make any other context switches, it'll either switch to one of the three currently running threads and run those or switch to a new instance of ThreadA which has to wait for the mutex to unlock before it can enter the critical section. Therefore, the maximum number of threads that can be in the critical section for this problem is 3. 

{\bf 5 [25 pts]} Consider a multithreaded web crawling and indexing program, which needs to first download a web page and then parse the HTML of that page to extract links and other useful information from it. The problem is both downloading a page and parsing it can be very slow depending on the content. Your goal is to make both these components as fast as possible. First, to speed up downloading, you delegate the task to m downloader threads, each with only a portion of the page to download. (Note that this is quite common in real life and a typical web browser does this all the time as long as the server supports this feature. Usually it is done through opening multiple TCP connections with the server and downloading equal sized chunks through each connection). The M chunks are downloaded to a single page buffer. Once all the chunks are downloaded into the buffer, you can then start parsing it. However, since you want to speed up parsing as well, you now use n parsing threads who again can parse the page independently, and together they take much less time.   

By now, you probably see that M download threads are acting as Producers and N parser threads as Consumers. Additionally, note that the both downloader and parser threads come from a pool of M Producer threads and N Consumer threads where M>m and N>n. Out of many of these, you have to let exactly m Producer threads carry out the download and then exactly n consumer threads parse, and then the whole cycle will repeat. You cannot assume m=M or n=N. Assume that you can call the function download(URL) to download the page and parse (chunk) to parse a chunk of the page. No need to be any more specific/concrete than that. The main thing of interest is the Producer-Consumer relation.  

Look at the given program 1PNC.cpp that works for 1 Producer and n Consumer threads that launches hundreds of producer and consumer threads. You need to extend the program such that it works for m-producers instead of just 1. Add necessary semaphores to the program. However, you will lose points if you add unnecessary Semaphores or Mutexes. To keep things simple, declare the mutexes as instances of the semaphore class given in  Semaphore.h. Test your program to make sure that it is correct. Include a file called Q5.cpp in your submission that contains the correct program.

Here is the expected output with m=3 and n=5:

\noindent
Producer [1] left buffer=1\\
Producer [3] left buffer=2\\
Producer [2] left buffer=3\\
$>>>>>>>>>>>>>>>>>>>>$Consumer [1] got $<<<<<<<<<<$3\\
$>>>>>>>>>>>>>>>>>>>>$Consumer [2] got $<<<<<<<<<<$3\\
$>>>>>>>>>>>>>>>>>>>>$Consumer [5] got $<<<<<<<<<<$3\\
$>>>>>>>>>>>>>>>>>>>>$Consumer [3] got $<<<<<<<<<<$3\\
$>>>>>>>>>>>>>>>>>>>>$Consumer [4] got $<<<<<<<<<<$3\\
Producer [5] left buffer=4\\
Producer [3] left buffer=5\\
Producer [6] left buffer=6\\
$>>>>>>>>>>>>>>>>>>>>$Consumer [6] got $<<<<<<<<<<$6\\
$>>>>>>>>>>>>>>>>>>>>$Consumer [7] got $<<<<<<<<<<$6\\
$>>>>>>>>>>>>>>>>>>>>$Consumer [8] got $<<<<<<<<<<$6\\
$>>>>>>>>>>>>>>>>>>>>$Consumer [10] got $<<<<<<<<<<$6\\
$>>>>>>>>>>>>>>>>>>>>$Consumer [9] got $<<<<<<<<<<$6\\
// ETC.\\

{\bf 6 [15 pts]}

{\bf 7 [20 pts]}

\end{document}