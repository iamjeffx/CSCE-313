\documentclass[12pt]{article}

\usepackage{times}
\usepackage{amsmath}
\usepackage{latexsym}
\usepackage{fullpage}
\usepackage{graphicx}
\usepackage{amsfonts}

\graphicspath{ {./images/} }

\newcommand{\NOT}{\neg}
\newcommand{\AND}{\wedge}
\newcommand{\OR}{\vee}
\newcommand{\XOR}{\oplus}
\newcommand{\IMPLIES}{\rightarrow}
\newcommand{\IFF}{\leftrightarrow}
\newcommand{\E}{\exists}
\newcommand{\A}{\forall}

\setlength{\parskip}{.1in}

\renewcommand{\baselinestretch}{1.1}

\begin{document}

\begin{center}

{\bf
CSCE 313\\
PA6-Report\\
Jeffrey Xu\\
11/23/20\\
}

\end{center}

\section{Time Complexity of TCP Connections}

We need to analyze the runtime and time complexity of our PA6 with previous runtimes of PAs. The main difference with this PA and other PAs is that now we are using TCP connections to make patient and file requests. Since this isn't running in the same machine per se, the runtimes can vary dramatically since TCP connections may have different reachabilities and connection speeds. This will give some longer runtimes for some requests compared to just using ICP methods. The plots below give the runtimes for various $w$ and $b$ values for patient requests. 

\end{document}